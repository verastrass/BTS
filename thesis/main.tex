\input{preamble.tex}
\NewBibliographyString{langjapanese}
\NewBibliographyString{fromjapanese}

\begin{document}
\Intro
В~современном мире распределённые системы являются основой различных объектов, например, веб-сервисов и пиринговых сетей. Поскольку обмен сообщениями в~системе осуществляется по~ненадёжным каналам связи, сообщения, посылаемые системе или самой системой, могут просматриваться, перехватываться и подменяться. Это приводит к~неправильному функционированию системы, отказу отдельных компонент или системы в~целом. Поэтому в~распределённых системах поддержке безопасности уделяется особое внимание.

Для~повышения надёжности системы случайные и умышленные сбои в~системе интерпретируются как Византийские ошибки (в~терминах <<Задачи о~Византийских генералах>>). В~этом случае можно использовать соответствующие отказоустойчивые техники, которые защитят систему и от~случайных, и от~умышленных сбоев.

В~данной работе рассматривается Византийское пространство кортежей~--- открытая распределённая устойчивая к~Византийским ошибкам система, основанная на~пространстве кортежей без~распределённой памяти, в~которой процессы взаимодействуют путём~обмена сообщениями.


\pagebreak


\section{Теоретические основы}\label{sec:1}
\subsection{Распределённые вычислительные системы}\label{subsec:1}
Распределённая вычислительная система~--- это набор независимых компьютеров, реализующий параллельную обработку данных на~многих вычислительных узлах. С~точки зрения пользователя этот набор является единым механизмом, предоставляющим полный доступ к~ресурсам. Существует возможность добавления новых ресурсов, свойст и методов, возможность перераспределения ресурсов по~системе, но информация об~этих событиях скрыта от~пользователя \autocite{Tanenbaum}.

Распределённая система, рассматриваемая в~данной работе, представлена множеством из~n~серверов. Взаимодействие клиентов с~системой происходит с~помощью вспомогательной промежуточной инфраструктуры, как показано на~рисунке\,\ref{clser}.
\begin{figure}[H]
	\centering \includegraphics[width=0.7 \textwidth, height=0.5 \textwidth]{img/ClientServer}  \caption{Взаимодействие клиентов с~распределённой системой} \label{clser}
\end{figure}

Одной из~важнейших характеристик распределённых систем является отказоустойчивость. Отказоустойчивость~--- это свойство системы сохранять работоспособность в~том случае, если какие-либо составляющие её компоненты перестали правильно функционировать. Компоненты системы могут стать нероботоспособны по~различным причинам, например, из-за~технологических сбоев или атак безопасности.

Большинство современных распределённых систем имеют характеристики открытых систем. Открытая распределённая система предполагает использование служб, вызов которых требует стандартного синтаксиса и семантики. Такая система может иметь неизвестное количество ненадёжных и неоднородных участников, то~есть участникам не нужно быть активными одновременно (свойство разъединённости во~времени) и не обязательно знать друг о~друге (свойство разъединённости в~пространстве). Связь между~узлами распределённой системы является ненадёжной (может прерываться, что повлечёт за~собой потерю сообщений), обмен сообщениями может происходить не мгновенно, а с~существенной задержкой. Кроме~того, любой узел системы может отказать или быть выключен в~любой момент времени. Все эти факторы неизбежно приводят к~неправильному функционированию системы. Один из~способов улучшить её надёжность~--- это интерпретировать случайные или умышленные неполадки как~Византийские ошибки (в~терминах <<Задачи византийских генералов>>), тогда использование отказоустойчивых техник сможет сделать координационную составляющую системы отказоустойчивой и для~случайных сбоев, и для~умышленных вторжений.

\subsection{Задача византийских генералов}\label{subsec:2}
Задача византийских генералов~--- это задача взаимодействия нескольких удалённых абонентов, получивших сообщения из~одного центра, причём часть этих абонентов, в~том числе центр, могут быть предателями, то~есть могут посылать заведомо ложные сообщения с~целью дезинформирования. Нахождение решения задачи заключается в~выработке единой стратегии действий, которая будет являться выигрышной для~всех абонентов.

Формулировка задачи состоит в~следующем. Византийская армия представляет собой объединение некоторого числа легионов, каждым из~которых командует свой генерал, генералы подчиняются главнокомандующему армии Византии. Поскольку империя находится в~упадке, любой из~генералов и даже главнокомандующий могут быть заинтересованы в~поражении армии, то~есть являться предателями. Генералов, не заинтересованных в~поражении армии, будем называть верными. В~ночь перед~сражением каждый из~генералов получает от~главнокомандующего приказ о~действиях во~время сражения: атаковать или отступать. Таким образом, имеем три возможных исхода сражения:
\begin{itemize}
	\item Благоприятный исход: все генералы атакуют противника, что приведёт к~его уничтожению и победе Византии.
	\item Промежуточный исход: все генералы отступят, тогда противник не будет побеждён, но Византия сохранит свою армию.
	\item Неблагоприятный исход: некоторые генералы атакуют противника, некоторые отступят, тогда Византийская армии потерпит поражение.
\end{itemize}

Так~как главнокомандующий тоже может оказаться предателем, то генералам не следует доверять его приказам. Однако если каждый генерал будет действовать самостоятельно, независимо от~других генералов, то вероятность наступления боагоприятного исхода становится низкой. Таким образом, генералам следует обмениваться информацией между~собой для~того, чтобы прийти к~единому решению.

На~практике задача византийских генералов решается с~помощью алгоритмов консенсуса, ярким представителем которых является алгоритм Паксос, предложенный Лесли Лампортом.

\subsection{Paxos}\label{subsec:3}
Paxos~--- это алгоритм для~решения задачи консенсуса в~сети ненадёжных вычислителей. Компоненты распределённой системы можно разделить на~3 группы:
\begin{itemize}
	\item Заявитель (Proposer)~--- выдвигает <<предложения>> (какие-либо значения), которые либо принимаются, либо отвергаются в~результате работы алгоритма консенсуса.
	\item Акцептор (Acceptor)~--- принимает или отвергает <<предложение>> Заявителя, согласует своё решение с~остальными Акцепторами, уведомляет о~своём решении Узнающих. Если Акцепторами было принято какое-либо значение, предложенное Заявителем, то оно называется утверждённым.
	\item Узнающий (Learner)~--- запоминает решения Акцепторов, принятые в~результате работы алгоритма консенсуса.
\end{itemize}

Компоненты распределённой системы могут принадлежать сразу нескольким группам, описанным выше, и вести себя и как Заявитель, и как Акцептор, и как Узнающий.

Такое распределение ролей в~системе гарантирует следующее:
\begin{itemize}
	\item Только предложенное Заявителем значение может быть утверждено Акцепторами.
	\item Акцепторами утверждается только одно значение из~всех предложенных Заявителями значений (возможно, противоречивых).
	\item Узнающий не сможет узнать об~утверждении какого-либо значения вплоть до~того момента, пока оно действительно не будет утверждено.
\end{itemize}




\subsection{Пространство кортежей} \label{tuplespace}
Кортеж~--- это структура данных, представляющая собой неизменяемый список фиксированной длины, элементы которого могут относиться к~различным типам данных. Два кортежа считаются равными, если совпадают их длины, а также типы и значения соответствующих полей.

Хранилище кортежей, в~котором доступ к~элементам может осуществляться параллельно, называется пространством кортежей. Оно является основой языка программирования Linda. Данный язык предназначен для~построения эффективных параллельных программ. Он включает в~себя три операции манипулирования данными (кортежами):
\begin{itemize}
	\item out - запись кортежа в~пространство кортежей.
	\item rd - недеструктивное чтение кортежа.
	\item in - деструктивное чтение (извлечение) кортежа.
\end{itemize}

Шаблоном кортежа будем называть кортеж, некоторые поля которого неопределены и не представляют важности. Кортеж соответствует шаблону, если длина кортежа равна длине шаблона и определённые в~шаблоне поля совпадают по~типу и значению с~соответствующими полями кортежа. Операция записи out принимает в~качестве входного параметра кортеж, все поля которого определены. Операции чтения rd и in принимают в~качестве входного параметра шаблон кортежа, по~которому производится поиск соответствующих кортежей в~пространстве. Таким образом, пространство кортежей можно рассматривать как разновидность распределённой памяти: например, одна группа процессов записывает данные в~пространство кортежей, а другая группа процессов извлекает данные из~пространства и использует их в~своей дальнейшей работе.

Для~того, чтобы поиск кортежей в~пространстве кортежей занимал минимальное время, адресация в~нём осуществляется по~содержимому (например, с~помощью алгоритмов хэширования). Иными словами, пространство кортежей~--- реализация парадигмы ассоциативной памяти.

В~данной работе рассматривается распределённая система под~названием <<Византийское пространство кортежей>>. Основой системы является пространство кортежей, имеющее ряд особенностей, которые позволяют добиться отказоустойчивости.

\subsection{Византийское пространство кортежей} \label{bts} 



 




\if 0
% Если typeOfWork в SETUP.tex задан как 2 или 3, то начинать
% надо не с section (раздел), а с главы (chapter)
\section{Несколько примеров в~\LaTeX{}}
\label{sec:examples}
\subsection{Как вставлять листинги и рисунки}
Для крупных листингов есть два способа. Первый красивый, но в нём не допускается
кириллица (у вас может встречаться в комментариях и
печатаемых сообщениях), он представлен на листинге~\ref{list:hwbeauty}.
\begin{ListingEnv}[H]% буква H означает Here, ставим здесь,
% элементы, которые нежелательно разрывать обычно не ставят
% посреди страницы: вместо H используется t (top, сверху страницы),
% или b (bottom) или p (page, на отдельной странице)
\begin{lstlisting}
#include <iostream>
using namespace std;

int main()
{
    cout << "Hello, world" << endl;
    system("pause");
    return 0;
}
\end{lstlisting}
%следующую команду для генерации подписи можно опустить,
% хотя рекомендуется все специальные элементы (таблицы, рисунки,
% листинги) подписывать. Если подпись пропустить, листинг также не получит
% номера и на него не сошлёшься в будущем
\caption{Программа “Hello, world” на \protect\cpp}
% далее метка для ссылки:
\label{list:hwbeauty}
\end{ListingEnv}

Второй не такой красивый, но без ограничений (см.~листинг~\ref{list:hwplain}).
\begin{ListingEnv}[H]
\begin{Verb}

#include <iostream>
using namespace std;

int main()
{
    cout << "Привет, мир" << endl;
}
\end{Verb}
\caption{Программа “Hello, world” без подсветки}
\label{list:hwplain}
\end{ListingEnv}

Можно использовать первый для вставки небольших фрагментов
внутри текста, а второй для вставки полного
кода в приложении, если таковое имеется.

Если нужно вставить совсем короткий пример кода (одна или две строки), то выделение  линейками и нумерация может смотреться чересчур громоздко. В таких случаях можно использовать окружения \texttt{lstlisting} или \texttt{Verb} без \texttt{ListingEnv}. Приведём такой пример с указанием языка программирования, отличного от заданного по умолчанию:
\begin{lstlisting}[language=Haskell]
fibs = 0 : 1 : zipWith (+) fibs (tail fibs)
\end{lstlisting}
Такое решение~--- со вставкой нумерованных листингов покрупнее
и вставок без выделения для маленьких фрагментов~--- выбрано,
например, в книге Эндрю Таненбаума и Тодда Остина по архитектуре
компьютера~\autocite{TanAus2013} (см.~рис.~\ref{fig:tan-aus}).

Наконец, для оформления идентификаторов внутри строк
(функция \lstinline{main} и тому подобное) используется
\texttt{lstinline} или, самое простое, моноширинный текст
(\texttt{\textbackslash texttt}).

\begin{figure}[p]% p означает, что нужно выделить для рисунка
% отдельную страницу; применяется для больших рисунков
\centering
%Здесь могла быть ваша лягушка.
\includegraphics[width=\textwidth]{img/tan-aus.png}
\caption{\label{fig:tan-aus}Пример оформления листингов в~\autocite{TanAus2013}}
\end{figure}

Использовать внешние файлы (например, рисунки) можно и на \href{http://overleaf.com}{overleaf.com}: ищите кнопочку upload.

\subsection{Как оформить таблицу}

Для таблиц обычно используются окружения table и tabular --- см. таблицу~\ref{tab:widgets}. Внутри окружения tabular используются специальные команды пакета booktabs — они очень красивые; самое главное: использование вертикальных линеек считается моветоном.

\begin{table}
\centering
\caption{\label{tab:widgets}Подпись к таблице --- сверху}
\begin{tabular}{llr}
\toprule
\multicolumn{2}{c}{Item} \\
\cmidrule(r){1-2}
Животное  & Описание    & Цена (\$) \\
\midrule
Gnat      & per gram    & 13.65      \\
          & each        & 0.01       \\
Gnu       & stuffed     & 92.50      \\
Emu       & stuffed     & 33.33      \\
Armadillo & frozen      & 8.99       \\
\bottomrule
\end{tabular}
\end{table}

\subsection{Как набирать формулы}

\LaTeX{} is great at typesetting mathematics. Let $X_1, X_2, \ldots, X_n$ be a sequence of independent and identically distributed random variables with $\text{E}[X_i] = \mu$ and $\text{Var}[X_i] = \sigma^2 < \infty$, and let
$$S_n = \frac{X_1 + X_2 + \cdots + X_n}{n}
      = \frac{1}{n}\sum_{i}^{n} X_i$$
denote their mean. Then as $n$ approaches infinity, the random variables $\sqrt{n}(S_n - \mu)$ converge in distribution to a normal $\mathcal{N}(0, \sigma^2)$.

\subsection{Как оформлять списки}

Нумерованные списки (окружение enumerate, команды item)…

\begin{enumerate}
  \item Like this,
  \item and like this.
\end{enumerate}

\dots маркированные списки \dots

\begin{itemize}
  \item Like this,
  \item and like this.
\end{itemize}

\dots списки-описания \dots

\begin{description}
  \item[Word] Definition
  \item[Concept] Explanation
  \item[Idea] Text
\end{description}

\Conc

Помните, что на все пункты списка литературы должны быть ссылки. \LaTeX\ просто не добавит информацию об издании из bib"/файла, если на это издание нет ссылки в тексте. Часто студенты используют в работе  электронные ресурсы: в этом нет ничего зазорного при одном условии: при каждом заимствовании следует ставить соответствующую ссылку. В качестве примера приведём ссылку на сайт нашего института~\autocite{mmcs}.

Для дальнейшего изучения \LaTeX\ рекомендуем книгу Львовского~\autocite{Lvo2003}: она хорошо написана, хотя и несколько устарела.
Обычно стоит искать подсказки на
\href{http://tex.stackexchange.com/}{tex.stackexchange.com}, а также
читать документацию по установленным пакетам с помощью
команды
\begin{Verb}
texdoc имя_пакета
\end{Verb}
или на \href{http://ctan.org/}{ctan.org}.

% Печать списка литературы (библиографии)
\printbibliography[%{}
    heading=bibintoc%
    %,title=Библиография % если хочется это слово
]
% Файл со списком литературы: biblio.bib
% Подробно по оформлению библиографии:
% см. документацию к пакету biblatex-gost
% http://ctan.mirrorcatalogs.com/macros/latex/exptl/biblatex-contrib/biblatex-gost/doc/biblatex-gost.pdf
% и огромное количество примеров там же:
% http://mirror.macomnet.net/pub/CTAN/macros/latex/contrib/biblatex-contrib/biblatex-gost/doc/biblatex-gost-examples.pdf

\appendix
\ifthenelse{\value{worktype} > 1}{%
  \addtocontents{toc}{%
      \protect\renewcommand{\protect\cftchappresnum}{\appendixname\space}%
      \protect\addtolength{\protect\cftchapnumwidth}{\widthof{\appendixname\space{}} - \widthof{Глава }}%
  }%
}{
  \addtocontents{toc}{%
      \protect\renewcommand{\protect\cftsecpresnum}{\appendixname\space}%
      \protect\addtolength{\protect\cftsecnumwidth}{\widthof{\appendixname\space{}}}%
  }%
}

\section{Пример работы программы}

Здесь длинный листинг с примером работы.
\fi

\end{document}
